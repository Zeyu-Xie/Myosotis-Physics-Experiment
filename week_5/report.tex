\documentclass[12pt,a4paper]{amsart}
\usepackage[UTF8]{ctex}
\usepackage{preamble}


\title{阻尼振动和受迫振动物理实验}

\begin{document}

\maketitle

\section{实验目的}
\subsection{观测阻尼振动,学习测量振动系统基本参数的方法}
\subsection{研究受迫振动的幅频特性和相频特性,观察共振现象} 
\subsection{观测不同阻尼对受迫振动的影响}
\section{实验仪器}

\section{实验原理}
\subsection{有粘滞阻尼的阻尼振动}
对于弹簧与摆轮组成的振动系统(如图1所示),设摆轮转动惯量为$J$,粘滞阻尼的阻尼力矩大小定义为角速度$\frac{d\theta}{dt}$与阻尼力矩系数$\gamma$的乘积,弹簧劲度系数为$k$,弹簧的反抗力矩为$-k\theta$。忽略弹簧的等效转动惯量,则转角$\theta$的运动方程为

\[
J\frac{d^2\theta}{dt^2} + \gamma\frac{d\theta}{dt} + k\theta = 0 \quad (1)
\]

记$\omega_0$为无阻尼时自由振动的固有角频率,其值为$\omega_0 = \sqrt{\frac{k}{J}}$,定义阻尼系数$\beta$为$\beta = \frac{\gamma}{2J}$,则式(1)变为

\[
J\frac{d^2\theta}{dt^2} + 2\beta\omega_0\frac{d\theta}{dt} + \omega_0^2\theta = 0 \quad (2)
\]

对于弱阻尼即$2\beta\omega_0 \ll 1$的情况,阻尼振动运动方程(1)的解为

\[
\theta = \theta_i e^{-\beta t}\cos(\omega t + \phi) \quad (3)
\]

式中$\theta_i$为摆幅,$\phi$为初始相位。由上式可知,阻尼振动角频率为$\omega = \sqrt{\omega_0^2 - \beta^2}$,相应的阻尼振动周期为$T = \frac{2\pi}{\sqrt{\omega_0^2 - \beta^2}}$。


\section{实验内容}
\section{数据处理及结果}
\section{实验小结}
\section{原始数据记录}

% \cite{Abu-Zurayk2013}\cite{Gerritsma2008}\cite{Gerritsma2008}\cite{Baseski2014}

\appendix


\bibliographystyle{unsrt}
{\footnotesize\bibliography{./library}}


\end{document}
